\documentclass[10pt]{article}

%%Fichier de configuration perso
\usepackage{maConfiguration}

\title{%
    \Huge
    \includegraphics[scale=.2]{mb4}\\[2cm]
    Les objets connectés pour enseigner\\
    l'algorithmique\\
    en lycée professionnel}
\author{%
    Groupe InEFLP\\
    \href{http://url.univ-irem.fr/ineflp}{\includegraphics[height=2cm]{fig-logo-ineflp}}\\
    de l'IREM\footnote{Institut de Recherche sur l'Enseignement des Mathématiques} de Marseille
    }
\date{\today}



\begin{document}



%%% Titre
\pagecolor{orangeamu!25}
\maketitle\thispagestyle{empty}


%%% Sommaire
\newpage\pagecolor{orangeamu!10}
\tableofcontents


%%% À propos de la brochure
\newpage\pagecolor{bleuamu!10}
\input{apropos.tex}


%%% Généralités
\newpage\pagecolor{bleuamu!10}
\style{plain}

\section{À propos de la carte \mb}


\mb est un microcontrôleur développé au Royaume-Unis.
Par ses caractéristiques techniques et ses interfaces
pédagogiques, cet objet possède un fort potentiel pour
l’enseignement de l’algorithmique.



\begin{wrapfigure}{r}{4cm}
	\includegraphics[width=\linewidth]{res/mb-ap-01.png}
	\legend{Carte \mb}
\end{wrapfigure}

Après un bref rappel historique, nous expliquerons plus en détail les caractéristiques propres de cet objet. Nous mettrons ensuite en avant la facilité de mise en œuvre en formation puis nous poursuivrons en donnant un premier aperçu de l’intérêt pédagogique de \mb.

\subsection{Bref historique}


Le développement de \mb s’inscrit dans le cadre d’une politique volontariste de développement de l’apprentissage de la programmation. L’objectif premier visait à équiper tous les élèves de 11/12 ans du Royaume-Unis ainsi que leur enseignant. Maintenant que c’est chose faite, le reste du monde peut en profiter aussi.

% \begin{wrapfigure}[12]{r}{5cm}
%     \includegraphics[width=\linewidth]{res/mb-ap-03}
%     \legend{Cartes \mb branchée en USB}
% \end{wrapfigure}


La BBC\footnote{Make It Digital - The BBC micro:bit. (s. d.). Consulté 29 mars 2017, à l’adresse \url{http://www.bbc.co.uk/programmes/articles/4hVG2Br1W1LKCmw8nSm9WnQ/the-bbc-micro-bit}} est le moteur de ce projet. 30 ans après sa première distribution d’ordinateurs aux enfants britanniques\footnote{BBC Micro. (2016, septembre 20). In Wikipédia. Consulté à l’adresse \url{https://fr.wikipedia.org/w/index.php?title=BBC_Micro&oldid=129763631}}, “la Vieille Dame” remet ça aujourd’hui. La BBC utilise ses moyens de diffusions pour promouvoir et accompagner les utilisateurs, notamment en proposant des émissions de TV dédié à cet objet sur un mode ludique et divertissant. Sur les 29\footnote{Partners. (s. d.). Consulté 29 mars 2017, à l’adresse \url{https://www.microbit.co.uk/partners}} partenaires de ce projet, se trouvent entre autres Microsoft\footnote{The BBC micro:bit and Microsoft - Microsoft Research. (s. d.). Consulté 29 mars 2017, à l’adresse \url{https://www.microsoft.com/en-us/research/project/the-bbc-microbit-and-microsoft/}} pour une partie logiciel et interface de programmation, ARM\footnote{Ltd, A. R. M. (s. d.). ARM | Innovation Hub - BBC micro:bit. Consulté 29 mars 2017, à l’adresse \url{http://www.arm.com/innovation/products/microbit.php}} pour la construction des processeur et la partie matériel, et Samsung\footnote{Code on the go with Samsung \& micro:bit. (s. d.). Consulté 29 mars 2017, à l’adresse \url{http://www.samsung.com/uk/citizenship/bbcmicrobit.html}} pour un support mobile. C’est donc un projet qui mobilise des acteurs majeurs du numériques et de la communication, prévu pour durer.

% \begin{figure}
%     \center
%     \includegraphics[width=0.5\linewidth]{res/mb-ap-02}
%     \legend{Cartes \mb lors d'un escape game utilisé en formation}
% \end{figure}


\subsection{La carte \mb}

Concrètement de quoi s’agit-il ? On parle ici de microcontrôleur, à savoir une carte électronique programmable pour interagir avec le monde réel.
C’est une version accessible de l’électronique que tout un chacun manipule au quotidien sans se poser de question, par exemple les dispositifs de domotique qui permettent de gérer à distance le chauffage, la sécurité, l’arrosage du géranium… Ou bien plus simplement la bouilloire programmable au degré ${}^{\circ}$C près, la guirlande du sapin qui clignote au rythme de Jingle Bells. 
Ce microcontrôleur permet d’élaborer par exemple un podomètre, un doudou sensoriel, un sismographe rudimentaire\ldots

\begin{figure}
    \centering
    \includegraphics[width=0.49\linewidth]{res/mb-ap-04}
    \hfill
    \includegraphics[width=0.49\linewidth]{res/mb-ap-05}
    \legend{Détails des entrées/sorties d'une carte \mb}
\end{figure}

L’interface de programmation est conçue pour être utilisable par un enfant d’une dizaine d’année, c’est donc la simplicité qui prime. On dispose en première approche d’une application internet utilisant le principe de la programmation par bloc, à savoir sur le principe des Blockly que l’on retrouve dans Scratch ou StudioCode. En plus d’une programmation accessible, l’interface propose une simulation de la carte. Ceci permet de voir directement les effets du programme dans l’interface. Pour un usage plus avancé il est notamment possible de programmer avec le langage Python\footnote{Python editor. (s. d.). Consulté 29 mars 2017, à l’adresse \url{http://python.microbit.org/editor.html}} ou Javascript.


\begin{minipage}[b]{0.45\linewidth}
    \vspace{0cm}
    \includegraphics[width=\linewidth]{res/mb-ap-07}
    \legend{Cartes \mb qui fait de la musique (stage 2018)}
\end{minipage}
\hfill
\begin{minipage}[b]{0.45\linewidth}
    Bien entendu de nombreux exemples de projets existent, qu’ils soient issus des émissions BBC ou de la communauté éducative. Sur le site officiel on trouve des idées, des tutoriels, des leçons\footnote{Idées | micro:bit. (s. d.). Consulté 29 mars 2017, à l’adresse \url{http://microbit.org/fr/ideas/}} comme par exemple : une alarme de trousse, un compteur de frappe (au baseball) ou encore des leçons sur l’accélération.
    \vspace{1em}
    \begin{center}
        \includegraphics[width=\linewidth]{res/mb-ap-08}
    \end{center}
\end{minipage}


\subsection{Programmer la carte \mb \emph{par blocs}}

\subsubsection{Une interface en ligne}

L’interface de programmation par blocs a été développée en partenariat avec microsoft, elle se trouve en ligne à cette adresse: \url{https://makecode.microbit.org}

Il s’agit donc d’une page internet mais dont le code est mis en cache par le navigateur ce qui signifie qu’elle reste opérationnelle hors ligne.


\begin{remarque}
    À partir de chrome par exemple, il est possible de créer un raccourci sur le bureau.
\end{remarque}



\subsubsection{Un simulateur !}

Le très gros intérêt de cette interface consiste en son simulateur de carte qui permet d’avoir un aperçu du fonctionnement du programme avant même de le télécharger sur la carte.

\begin{figure}[h]
    \centering
    \includegraphics[width=0.75\linewidth]{mb-prog.png}
    \legend{Interface de programmation par bloc qui intègre un \emph{simulateur}}
\end{figure}




\begin{remarque}
    Le simulateur peut ne pas fonctionner hors ligne.
\end{remarque}

\subsubsection{Compilation et enregistrement}

Le téléchargement sur la carte se fait très simplement puiqu’elle est reconnue comme une clé USB. Il suffit donc de cliquer sur Télécharger et de copier le fichier obtenu (.hex) sur la carte.

\subsubsection{Programmation par bloc}

Comme toute interface de programmation par blocs, elle est très intuitive à manipuler. Les premiers programmes se font très simplement et les catégories sont classées par couleurs et par technicité.

\begin{remarque}
    L’interface propose aussi de programmer en javascript, il suffit juste de cliquer sur un bouton pour changer de type de programmation.
\end{remarque}

\subsubsection{Documentation}

\begin{itemize}
    \item Une page de documentation présente les éléments de base pour la programmation par blocs\\
        \url{https://makecode.microbit.org/blocks}
    \item Une page de références présentent quelques fonctionnalités propre au microbit\\
        \url{https://makecode.microbit.org/reference}
\end{itemize}


\subsection{Programmer la carte \mb en \emph{Python}}

Le \mb peut exécuter une version allégée de Python qui s’appelle MicroPython. C’est une version spécialement dédiée aux microcontroleurs.

\subsubsection{Une interface en ligne}

Il est possible de programmer en python à partir d’un éditeur en ligne \url{http://python.microbit.org}

L’interface est cependant assez pauvre en fonctionnalité et ne dispose pas de l’autocomplétion.

\begin{figure}[h]
    \centering
    \includegraphics[width=0.65\linewidth]{res/mb-prog2.png}
    \legend{Interface de programmation en ligne}
\end{figure}

\subsubsection{Mu : une interface complète}

Comme le dit (en anglais) la page d’accueil de Mu : Mu est un éditeur de code simple pour les programmeurs débutants. Il est développé en Python et fonctionne sur Windows, OSX, Linux et Raspberry Pi.


\begin{figure}[h]
    \centering
    \includegraphics[width=0.65\linewidth]{res/mb-prog3.png}
    \legend{Interface de programmation Mu}
\end{figure}

\subsubsection{Programmation}

L’autocomplétion et l’autoindentation est très efficace. L’interface est rapidement utilisable par un débutant en programmation.
Compilation et enregistrement

Le téléchargement sur la carte se fait très simplement puisqu’il suffit de cliquer sur le bouton Flash . Il est tout de même préférable d’avoir au préalable le réflexe de vérifier le code avec Check .

\subsubsection{Communication série}

La fonction REPL de Mu permet d’ouvrir une communication via un port série avec le \mb. Il est ainsi possible d’envoyer et de recevoir des données. Sur les versions bêta il y a même un plotteur qui permet de visualiser graphiquement les données reçues.

\subsubsection{Documentation}

Il est existe un documentation sur microbit et micropython, qui bien qu’en anglais reste très accessible.

\url{https://microbit-micropython.readthedocs.io/}




%%% Fiches
\newpage\nopagecolor\input{mb-borneSatisfaction.tex}
\newpage\nopagecolor\input{mb-fraction.tex}
\newpage\nopagecolor\input{mb-dede.tex}
\newpage\nopagecolor\input{mb-truque.tex}
\newpage\nopagecolor\input{mb-pile.tex}
\newpage\nopagecolor\style{mb} %pour microbit


\section{Fluctuation d'échantillonnage avec \mb}


\subsection{Description}

\subsubsection{Objectif}


%   bloc de formule
%   sans titre et fond bleu cyan
\begin{formule}
Le but de ce projet est d'expérimenter la fluctuation d'échantillonnage à  partir d'une situation classique, en établissant tout d'abord un modèle d'expérience aléatoire à partir des données de la situation, puis en proposant de programmer le tirage d'échantillons pour une taille n fixée.
\end{formule}


\subsubsection{Intérêt}

 L'utilisation de l'interface \mb permet d'obtenir rapidement un programme fonctionnel, mais aussi d'afficher graphiquement la série de données produites. Cela représente un intérêt non-négligeable lorsque l'on souhaite traiter la fluctuation d'échantillonnage.

%liste d'arguments
\begin{description}
    \item [Simulation d'une grande série d'expériences aléatoires] Contrairement à l'usage du tableur, où l'élève va devoir manipuler un grand nombre de données en colonnes et en lignes, au risque de se perdre dans leur traitement, l'approche algorithmique de  ce problème permet d'aller à l'essentiel.
    \item [Afficher les données]
    En utilisant le navigateur chrome, il est possible de recueillir les données générées et de les visualiser. L'échellle du graphique se réglant automatiquement sur le maximum et le minimum des données, cela permet une compréhension immédiate du phénomène.
\end{description}


\subsubsection{Matériel}
\begin{itemize}
%   matériel pour micro:bit
    \item 1 $\times$ \matosMb \emph{(facultatif car le simulateur peut suffire)}
%   site pour micro:bit
    \item 1 $\times$ accès internet : IDE programmation par bloc \url{http://makecode.microbit.org/}
    \item lien vers l'activité 1 : \url{https://makecode.microbit.org/_TbPFTK8eaKes}
    \item lien vers l'activité 2 : \url{https://makecode.microbit.org/_PW8LCg82z3fh}
    \item lien vers l'activité 3 :
    \url{https://makecode.microbit.org/_11aUTkWzR60J}
\end{itemize}

\newpage

\subsubsection{Progression proposée}


%   bloc méthode
%   titre + fond bleu
\begin{methode}
    On propose ici d'aborder la problématique en trois temps :

    \begin{enumerate}
        \item \textbf{Prise en main - Vérifier le modèle.} \\
            Pour faciliter la prise en main et gagner du temps on propose aux élèves de vérifier un code déjà prêt. Cela facilite l'appropriation du problème.
        \item \textbf{Intermédiaire - Du modèle à la génération d'échantillons.}\\
            À partir du modèle, l'élève doit élaborer un algorithme afin de produire des échantillons de taille fixé.
        \item \textbf{Avancé - Visualisation des données}\\
            Cette étape consiste en une amélioration du programme précédent afin de pouvoir visualiser les données issues de la simulation.

    \end{enumerate}
\end{methode}

%
% activité de niveau 1
%

%   saut de page
\newpage

%   titre de la sous section
\subsection{Niveau prise en main - Vérifier le modèle\ldots}

\subsubsection{Activité élève}

% commande perso \CARTOUCHE
%   5 paramètres :
%       * durée
%       * public
%       * travail en maths
%       * travail en sciences
%       * travail en algo
\cartouche
{0,3 h}         %durée
{première}           %public
{statistiques et probabilités}        %maths
{}     %sciences
{instruction conditionnelle}       %algo


%   petite image de logo qui va
%   se mettre dans le bloc élève
\begin{wrapfigure}{r}{3cm}
    \includegraphics[width=\linewidth]{res/mb-fluctuations-illustration.png}
\end{wrapfigure}

%   bloc élève
%   fond orange
\begin{eleve}
    \texttt{\textsc{Ta Mission} : Utiliser \mb pour simuler des \emph{naissances}!}

    La situation est la suivante : à Ufa, en Russie, 51,2 \%  des naissances sont des garçons.

    Dans cette ville, une usine agrochimique expose ses employés à des pesticides contenant de la dioxine.

    D’après une étude de l’université de Montréal, parmi les 227 enfants nés d’un parent travaillant dans cette usine, 91 sont des garçons.

    Cette étude cherche à déterminer si l’usine interfère sur les naissances.

    Pour simuler les naissances à Ufa, on propose d'utiliser le programme ci-dessous.

    \emph{Expliquer pourquoi} ce programme modélise correctement les naissance à Ufa.

    \emph{Proposer} une façon de l'exploiter afin de vérifier l'influence des produits chimiques sur les naissances.

%   ajout d'une image
    \includegraphics[width=0.5\linewidth]{res/mb-fluctuation-activite1.png}

\end{eleve}



\subsubsection{Notes pour l'enseignant}

%
%   méthode et remarque
%
\begin{methode}
Dans cette activité, il s'agit de vérifier que les élèves ont bien compris la problématique et notamment comment est utilisé la fréquence d'apparition du caractère "garçon".

Bien entendu il faut suggérer aux élèves de tester le programme, afin d'en appréhender les limites.
\end{methode}


\begin{remarque}
    Ici l'utilisation de la variable \emph{p} n'est pas indispensable pour l'algorithme. Son intérêt est pédagogique : elle permet de mettre en évidence la donnée utilisée ainsi que de de faire le lien avec le vocabulaire et les notations utilisées dans le cours.

   Avant de passer à l'activité 2, il peut être préférable de lister avec les élèves les éléments manquants qui permettraient de produire et de traiter un échantillon comparable à celui de l'étude :
   \begin{itemize}
       \item une boucle répéter afin de produire un échantillon de taille 227
       \item des variables pour dénombrer les naissances de garçons (et de filles ?)
   \end{itemize}
\end{remarque}

%
% activité de niveau 2
%

%   saut de page
\newpage

%   titre de la sous section
\subsection{Niveau intermédiaire - Générer des échantillons\ldots}

\subsubsection{Activité élève}

% commande perso \CARTOUCHE
%   5 paramètres :
%       * durée
%       * public
%       * travail en maths
%       * travail en sciences
%       * travail en algo
\cartouche
{0,5 h}         %durée
{première}           %public
{statistiques et probabilités}        %maths
{}     %sciences
{instruction conditionnelle; boucle}       %algo


%   petite image de logo qui va
%   se mettre dans le bloc élève
\begin{wrapfigure}{r}{3cm}
    \includegraphics[width=\linewidth]{res/mb-fluctuations-illustration.png}
\end{wrapfigure}

%   bloc élève
%   fond orange
\begin{eleve}
    \texttt{\textsc{Ta Mission} : Utiliser \mb pour simuler des \emph{naissances}!}

    Utilise les blocs proposés afin de générer des échantillons de taille identique à celui de l'étude.

    \emph{Construire} un programme qui affiche le nombre de garçons obtenus dans un échantillons de 227 naissances.

    \emph{Utiliser} le programme afin de vérifier si la situation de la problématique est vraisemblable.

%   ajout d'une image
    \includegraphics[width=0.8\linewidth]{res/mb-fluctuation-activite2-blocs.png}

\end{eleve}

%   saut de page
\newpage

\subsubsection{Notes pour l'enseignant}

%
%   méthode et remarque
%
\begin{methode}
Dans cette activité, il s'agit de vérifier que les élèves se sont bien appropriés la problématique, notamment par rapport à la taille de l'échantillon.

Bien entendu il faut suggérer aux élèves de tester le programme plusieurs fois.

Le programme attendu peut être celui-ci si l'élève a utilisé tous les blocs proposés (voir \emph{Remarque} ci-dessous) :

%   ajout d'une image
    \includegraphics[width=0.8\linewidth]{res/mb-fluctuation-activite2-proposition.png}

\end{methode}


\begin{remarque}
    Ici l'utilisation de la variable \emph{f} n'est pas indispensable pour l'algorithme. Son intérêt est pédagogique : elle permet de faire le lien avec l'activité précédente en conservant le modèle proposé.

   Avant de passer à l'activité 3, il est tout de même préférable de faire s'interroger les élèves sur la nécessité de l'existence de la variable \emph{f}.

   Enfin, on interrogera les élèves sur le nombre d'échantillons qu'ils jugent nécessaires afin de valider leurs hypothèses.
\end{remarque}


%
% activité de niveau 3
%

%   saut de page
\newpage

%   titre de la sous section
\subsection{Niveau avancée - Produire des données\ldots}

\subsubsection{Activité élève}

% commande perso \CARTOUCHE
%   5 paramètres :
%       * durée
%       * public
%       * travail en maths
%       * travail en sciences
%       * travail en algo
\cartouche
{0,5 h}         %durée
{première}           %public
{statistiques et probabilités}        %maths
{}     %sciences
{boucles imbriquées; communication}       %algo


%   petite image de logo qui va
%   se mettre dans le bloc élève
\begin{wrapfigure}{r}{3cm}
    \includegraphics[width=\linewidth]{res/mb-fluctuation-activite3-illus.png}
\end{wrapfigure}

%   bloc élève
%   fond orange
\begin{eleve}
    \texttt{\textsc{Ta Mission} : Utiliser \mb pour simuler des \emph{naissances} et produire des \emph{données}!}

    Utilise les blocs proposés afin de générer 100 échantillons de taille identique à celui de l'étude.

    \emph{Construire} un programme qui envoie une série de 100 valeurs, chacune correspondant au nombre de garçon dans un échantillon.

    \emph{Utiliser} le programme et la fonctionnalité \emph{Afficher la console} du simulateur pour visualiser les données.

    \emph{Exploiter} les données produites pour déterminer si l'usine interfère sur les naissances de garçons.

%   ajout d'une image
    \includegraphics[width=0.8\linewidth]{res/mb-fluctuation-activite3-blocs.png}

\end{eleve}

%   saut de page
\newpage

\subsubsection{Notes pour l'enseignant}

%
%   méthode et remarque
%
\begin{methode}
Dans cette activité, il s'agit de vérifier que les élèves sont bien capable d'interpréter le graphique obtenu.

Bien entendu il faut suggérer aux élèves de tester le programme plusieurs fois, en prenant soin d'attendre la fin d'une série avant d'en lancer une deuxième

Le programme attendu peut être celui-ci  :

%   ajout d'une image
    \includegraphics[width=0.8\linewidth]{res/mb-fluctuation-activite3-proposition.png}

\end{methode}


\begin{remarque}
   On interrogera les élèves sur les minimum et les maximum observés et sur la fréquence d'apparition d'un effectif inférieur ou égal à 91.

   Les données produites sont exportables en .csv et donc exploitables dans un tableur.
\end{remarque}

\newpage\nopagecolor\input{mb-oscillations.tex}
\newpage\nopagecolor\style{ft}

\section{Mémo Mu-Editor pour \mbpy}



\begin{methode}[1er programme]
	\begin{multicols}{2}
		Afficher un premier texte sur l'écran \mb.\\	
		\textbf{Connecter} la carte à l'ordinateur
		\\[1em]
		
		\includegraphics[height=10em]{res/mu/005.jpg}
		\includegraphics[height=10em]{res/mu/006.jpg}
		
	\columnbreak


	\textbf{Ouvrir} Mu-editor\\
	\textbf{Copier} le code ci-dessous.
	\begin{mucode}
from microbit import *
display.scroll("Hello, World!")
	\end{mucode}
	\textbf{Flasher} la carte {\tiny(envoyer le programme dans 
	la carte)}
	\hfill \includegraphics[width=3em,valign=c]{res/flash.png}
	\end{multicols}
\end{methode}

	

\begin{minipage}[t]{0.5\linewidth}
\begin{methode}[Des images]\rule{-0.25em}{1.6em}
		Afficher une image sur l'écran du \mb.
		\\[1em]

		\textbf{Copier} le code et 
		\textbf{flasher} la carte
		\hfill\includegraphics[width=3em,valign=c]{res/flash.png}

\begin{mucode}
from microbit import *
for loop in range (10):
	display.show(Image.HEART)
	sleep(100)
	display.show(Image.HEART_SMALL)
	sleep(50)
display.scroll("Hello, World!")
\end{mucode}

	\begin{center}
		\includegraphics[width=8.0em]{res/mbpy-init-heart.png}
	\end{center}
\end{methode}
\end{minipage}
%
%
\begin{minipage}[t]{0.5\linewidth}
\begin{methode}[Les mouvements]\rule{-0.25em}{1.6em}
	Afficher des enregistrements de \textbf{l'accéléromètre}
	\\[1em]

	\textbf{Copier} le code et
	\textbf{flasher} la carte
	\hfill\includegraphics[width=3em,valign=c]{res/flash.png}\\
	\begin{mucode}
from microbit import *
display.show(Image.YES)
while True:
    valeurs= accelerometer.get_values()
    print (valeurs)
    sleep(100)
	\end{mucode}

	Afficher la vue \textsc{repl} (\textbf{terminal série})
	\hfill\includegraphics[width=3em,valign=c]{res/ft_repl.png}\\
	\textbf{Réinitialiser} le \mb
	\hfill\includegraphics[width=7em,valign=t]{res/mu/060.png}

\end{methode}
\end{minipage}


\begin{remarque}
	Vous devriez avoir un affichage du type\\
		\hfill
		\includegraphics[height=18em,valign=t]{res/mu/050.png}
		\includegraphics[height=18em,valign=t]{res/mu/070.png}
		\hfill	
\end{remarque}



%  MÉTHODE DES GRAPHIQUES
\begin{minipage}[t]{0.5\linewidth}
\begin{methode}[Graphique]\rule{-0.25em}{1.6em}
	Afficher les enregistrements de l'accéléromètre sous forme de \textbf{graphique}
	\\[1em]

	% \begin{multicols}{2}

		\textbf{Copier} le code et 	
		\textbf{flasher} la carte 
		\hfill\includegraphics[width=3em,valign=c]{res/flash.png}

		\begin{mucode}
from microbit import *
display.show(Image.YES)
while True:
	valeurs= accelerometer.get_values()
	print (valeurs)
	sleep(100)
		\end{mucode}
				
		Afficher la vue \textbf{Graphique}
		\hfill\includegraphics[width=3em,valign=c]{res/plotter.png}\\
		\textbf{Réinitialiser} le \mb
		\hfill\includegraphics[width=7em,valign=t]{res/mu/060.png}

	% \columnbreak

	% \end{multicols}

\end{methode}
\end{minipage}
%
%
% REMARQUE 
\begin{minipage}[t]{0.5\linewidth}
\begin{remarque}\rule{-0.25em}{1.6em}
	Vous devriez avoir un affichage du type\\
	
	\begin{center}
		
		\includegraphics[width=\linewidth,valign=t]{res/mu/090.png}	
	\end{center}
	
\end{remarque}
\end{minipage}


\begin{minipage}[t]{0.5\linewidth}
	\begin{methode}[Communication radio]\rule{-0.25em}{1.6em}

	\rule{5em}{30em}

	\end{methode}
\end{minipage}
%
%
\begin{minipage}[t]{0.5\linewidth}
	\begin{methode}[Enregistrer des fichiers]\rule{-0.25em}{1.6em}

	\rule{5em}{40em}

	\end{methode}
\end{minipage}
%
%
\newpage\nopagecolor\input{mbpy-FT_python.tex}
\newpage\nopagecolor\style{mb}

%   Titre de la sous section
\section{Mémo : les blocs principaux \mb}


%   colonne de gauche
\begin{minipage}[t]{0.75\linewidth}

    \begin{blocBase}\\
      \rule{-0.25em}{2em}
      Dans cette catégorie, on trouve les deux blocs essentiels qui sont d'ailleurs présents par défaut à l'ouverture d'un nouveau projet.

      \begin{itemize}
        \item Le bloc "au démarrage" permet d'initialiser un programme, la séquence d'instruction qui y sera placée ne sera donc exécuté qu'une seule fois.
        \item   Le bloc "toujours" contient la séquence d'instruction qui sera perpétuellement exécutée par le processeur.
        \item Outre ces blocs, vous trouverez dans cette catégorie les éléments permettant d'afficher des figures, du texte, d'effacer l'écran, de faire une pause, etc ...
      \end{itemize}

    \end{blocBase}

\end{minipage}
%   petit "ressort"
%   pour centrer les 2 colonnes
\hfill
%   colonne de droite
\begin{minipage}[t]{0.25\linewidth}~\\
\vspace{5mm}

  	\includegraphics[scale=0.4]{res/blocsMkCd/MB_makecode_audemarrage.png}\\[0.5em]
  	\includegraphics[scale=0.4]{res/blocsMkCd/MB_makecode_toujours.png}\\[0.5em]
    \includegraphics[scale=0.4]{res/blocsMkCd/MB_makecode_base-icone.png}\\[0.5em]
    \includegraphics[scale=0.4]{res/blocsMkCd/MB_makecode_base-texte.png}\\[0.5em]
    \includegraphics[scale=0.4]{res/blocsMkCd/MB_makecode_base-pause.png}


\end{minipage}



%%%% Entrées
%   colonne de gauche
\begin{minipage}[t]{0.75\linewidth}

    \begin{blocEntrees}\\
      \rule{-0.25em}{2em}
La catégorie "Entrées" proposent notamment des blocs "Lorsque" qui fonctionnent de la même façon que le bloc "Toujours", à la différence que le \mb est ici en perpétuelle attente d'un évènement, que ce soit l'appui sur un bouton, un geste ou l'activation d'une broche.\\
\vspace{5mm}
On y trouve aussi les blocs permettant de recueillir les données :
\begin{itemize}
  \item savoir si un bouton est pressé;
  \item connaître la mesure de l'accélération;
  \item relever la température (du processeur);
\end{itemize}

    \end{blocEntrees}

\end{minipage}
%   petit "ressort"
%   pour centrer les 2 colonnes
\hfill
%   colonne de droite
\begin{minipage}[t]{0.25\linewidth}~\\
  \vspace{5mm}

    \includegraphics[scale=0.4]{res/blocsMkCd/MB_makecode_entrees-bouton.png}\\[0.5em]
    \includegraphics[scale=0.4]{res/blocsMkCd/MB_makecode_entrees-geste.png}\\[0.5em]
    \includegraphics[scale=0.4]{res/blocsMkCd/MB_makecode_entrees-broche.png}\\[0.5em]
    \includegraphics[scale=0.4]{res/blocsMkCd/MB_makecode_entrees-boutonPress.png}\\[0.5em]
    \includegraphics[scale=0.4]{res/blocsMkCd/MB_makecode_entrees-accel.png}\\[0.5em]
    \includegraphics[scale=0.4]{res/blocsMkCd/MB_makecode_entrees-temp.png}


\end{minipage}

%%%%

%%%%blocRadio
%   colonne de gauche
\begin{minipage}[t]{0.75\linewidth}

    \begin{blocRadio}\\
      \rule{-0.25em}{2em}
      Les fonctions de communication du \mb se trouve dans cette catégorie.\\
      Les données envoyées (nombre et/ou chaîne de caractères) ne sont pas "routées" : les reçoivent qui peut. Pour filtrer les réceptions il est possible de définir des groupes.\\
      \vspace{5mm}
      Les blocs de "réception" se comportent comme des blocs d'entrées et attendent perpétuellement un événement.


    \end{blocRadio}

\end{minipage}
%   petit "ressort"
%   pour centrer les 2 colonnes
\hfill
%   colonne de droite
\begin{minipage}[t]{0.25\linewidth}~\\
  \vspace{5mm}

    \includegraphics[scale=0.4]{res/blocsMkCd/MB_makecode_radio-envoyer.png}\\[0.5em]
    \includegraphics[scale=0.3]{res/blocsMkCd/MB_makecode_radio-recevoir.png}\\[0.5em]


\end{minipage}
%%%%

%%%% boucles
%   colonne de gauche
\begin{minipage}[t]{0.75\linewidth}

    \begin{blocBoucle}\\
      \rule{-0.25em}{2em}
      Cette catégorie contient les blocs permettant de faire des boucles ils sont aux nombres de 4 :

      \begin{itemize}
        \item "répéter", qui est le plus simple et qui convient dans un grand nombre de cas;
        \item  "tant que", qui répète la séquence tant que la condition indiquée est vraie;
        \item "pour" qui répète une séquence en incrémentant une valeur entière ou un index de liste;
      \end{itemize}

    \end{blocBoucle}

\end{minipage}
%   petit "ressort"
%   pour centrer les 2 colonnes
\hfill
%   colonne de droite
\begin{minipage}[t]{0.25\linewidth}~\\
  \vspace{5mm}


  	\includegraphics[scale=0.4]{res/blocsMkCd/MB_makecode_boucles-repeter.png}\\[0.5em]
  	\includegraphics[scale=0.4]{res/blocsMkCd/MB_makecode_boucles-tantque.png}\\[0.5em]
    \includegraphics[scale=0.4]{res/blocsMkCd/MB_makecode_boucles-parcourir.png}


\end{minipage}

%%%%

%%%%blocLogique
%   colonne de gauche
\begin{minipage}[t]{0.75\linewidth}

    \begin{blocLogique}\\
      \rule{-0.25em}{2em}
      Dans la catégorie "Logique" nous allons trouver les blocs relatifs aux instructions conditionnelles, aux tests et au booléens\\

      \begin{itemize}
        \item Les blocs conditionnels permettent d'exécuter une suite d'instruction si le résultat d'un test donné est vrai.On modifie le nombres de conditions en cliquant sur + ou -.
        \item Les blocs de test permettent de comparer des nombres, des chaînes de caractères, des booléens...
        \item Les blocs booléens permettent d'effectuer des opérations logiques.
      \end{itemize}

    \end{blocLogique}

\end{minipage}
%   petit "ressort"
%   pour centrer les 2 colonnes
\hfill
%   colonne de droite
\begin{minipage}[t]{0.25\linewidth}~\\
  \vspace{5mm}

    \includegraphics[scale=0.4]{res/blocsMkCd/MB_makecode_logique-sisinon.png}\\[0.5em]
    \includegraphics[scale=0.4]{res/blocsMkCd/MB_makecode_logique-test.png}\\[0.5em]
    \includegraphics[scale=0.4]{res/blocsMkCd/MB_makecode_logique-bool.png}

\end{minipage}
%%%%

%%%%blocVariable
%   colonne de gauche
\begin{minipage}[t]{0.75\linewidth}

    \begin{blocVariable}\\
      \rule{-0.25em}{2em}
      Par défaut, cette catégorie est vide tant qu'aucune variable n'a été définie, ou qu'aucun bloc comprenant une variable par défaut n'a été utiliser.\\
      Une variable peut bien sûr contenir un nombre, un booléen, une chaîne de caractères, une liste...

      \begin{itemize}
        \item Le bloc "définir" permet d'affecter une valeur à la variable.
        \item Le bloc "changer par" consisite à incrémenter la valeur de la variable
      \end{itemize}

    \end{blocVariable}

\end{minipage}
%   petit "ressort"
%   pour centrer les 2 colonnes
\hfill
%   colonne de droite
\begin{minipage}[t]{0.25\linewidth}~\\
  \vspace{5mm}

    \includegraphics[scale=0.4]{res/blocsMkCd/MB_makecode_variables-variable.png}\\[0.5em]
    \includegraphics[scale=0.4]{res/blocsMkCd/MB_makecode_variables-definir.png}\\[0.5em]
    \includegraphics[scale=0.4]{res/blocsMkCd/MB_makecode_variables-incrementer.png}

\end{minipage}
%%%%

%%%%blocMaths
%   colonne de gauche
\begin{minipage}[t]{0.75\linewidth}

    \begin{blocMaths}\\
      \rule{-0.25em}{2em}
      Vous trouverez dans cette catégorie tous les blocs pour les opérations classiques, mais aussi les blocs permettant de générer de l'aléatoire, ou encore de contraindre ou mapper des valeurs.


    \end{blocMaths}

\end{minipage}
%   petit "ressort"
%   pour centrer les 2 colonnes
\hfill
%   colonne de droite
\begin{minipage}[t]{0.25\linewidth}~\\
  \vspace{5mm}

    \includegraphics[scale=0.4]{res/blocsMkCd/MB_makecode_maths-aleaEntreBornes.png}\\[0.5em]
    \includegraphics[scale=0.4]{res/blocsMkCd/MB_makecode_maths-aleaVraiFaux.png}\\[0.5em]
    \includegraphics[scale=0.4]{res/blocsMkCd/MB_makecode_maths-mapper.png}\\[0.5em]


\end{minipage}
%%%%

%%%%blocFonctions
%   colonne de gauche
\begin{minipage}[t]{0.75\linewidth}

    \begin{blocFonctions}\\
      \rule{-0.25em}{2em}
      Pour accéder à cette catégorie, il faut dérouler le menu \includegraphics[scale=0.5]{res/blocsMkCd/MB_makecode_avance.png}.\\
      Tout comme pour la catégorie "Variables", il faut définir au préalable une fonction pour voir apparaître le bloc correspondant.\\
      \vspace{5mm}
      Une fonction peut être définie avec ou sans paramètres d'entrées.\\
      Une fonction en bloc ne renvoie pas de valeur.\\
      Une fois la fonction définie, on dispose d'un bloc pour l'appeler.


    \end{blocFonctions}

\end{minipage}
%   petit "ressort"
%   pour centrer les 2 colonnes
\hfill
%   colonne de droite
\begin{minipage}[t]{0.25\linewidth}~\\
  \vspace{5mm}

    \includegraphics[scale=0.4]{res/blocsMkCd/MB_makecode_fonctions-definir.png}\\[0.5em]
    \includegraphics[scale=0.4]{res/blocsMkCd/MB_makecode_fonctions-appeler.png}\\[0.5em]


\end{minipage}
%%%%

\newpage\nopagecolor\input{mb_makecode_1erProgramme.tex}
\newpage\nopagecolor\input{mbpy-initiationPython.tex}
\newpage\nopagecolor\input{mbpy-dede.tex}
\newpage\nopagecolor\input{mbpy-pilecom.tex}
\newpage\nopagecolor\input{maqueen-distance.tex}
\newpage\nopagecolor\input{mbot-vitesse.tex}
\newpage\nopagecolor\input{mbot-vitesse-rot.tex}
\newpage\nopagecolor\input{mbot-natureMouvement.tex}
\newpage\nopagecolor\input{st-truque.tex}
\newpage\nopagecolor\input{st-pile.tex}


\end{document}