\style{mbpy}

\section{Pile ou face communicant avec \mbpy}


\subsection{Description}

\subsubsection{Objectif}

\begin{formule}
Le but de ce projet est de simuler une expérience aléatoire de lancer de pièce truquée avec une carte \mb. et d'envoyer par radio l'issue afin de centraliser les résultats.
Cette activité permet de prendre en main facilement l'interface de développement en python
\end{formule}

\subsubsection{Intérêt}
Cette activité, idéale pour découvrir l'interface de programmation, propose d'utiliser python pour programmer une situation non-équiprobable et d'exploiter le module radio du \mb. 
\begin{description}
    \item [Production et traitement de données] Cette situation permet de travailler des notions de probabilités et de statistiques.
    \item [Programmation Python] Cette activité nécessite que les élèves aient déjà été initié à Python, sans pour autant demandé une grande maîtrise.
    
\end{description}


\subsubsection{Matériel}
\begin{itemize}
    \item 1 $\times$ \matosMb
    \item 1 $\times$ IDE programmation python (Mu) \url{https://codewith.mu/} ou interface de programmation ligne \url{https://python.microbit.org/v/1.1}
\end{itemize}

\subsubsection{Progression proposée}


%   bloc méthode
%   titre + fond bleu
\begin{methode}
    On propose ici d'aborder la problématique en deux temps :
    
    \begin{enumerate}
        \item \textbf{Tirage aléatoire et envoi des issues} \\
            Les élèves doivent modéliser une expérience aléatoire avec une probabilité donnée, et envoyer l'issue par radio.
        \item \textbf{Réception et traitement des données}\\
            L'incrémentation des variables et le traitement des données est déclenchée par la réception d'un message radio.
        
            
    \end{enumerate}
\end{methode}


%
% activité de niveau 1
%
\newpage
\subsection{Niveau simple}
\subsubsection{Activité élève}

% commande perso \CARTOUCHE
%   5 paramètres : 
%       * durée
%       * public
%       * travail en maths
%       * travail en sciences
%       * travail en algo
\cartouche{0,5 h}{2de}{expérience aléatoire}{}{fonction ; événement ; boucle}




\begin{eleve}
\texttt{\scshape{Mission : }
Programme \mb~pour simuler un lancer de \emph{Pile ou Face} truqué !}


On veut que notre \emph{pièce virtuelle} ait 55\% de chance de tomber sur "pile".


Le programme doit débuter avec les instructions ci-dessous afin :  
\begin{enumerate}
    \item d' importer les bibliothèques nécessaires ;
\pyfile{1}{3}{./res/mbpy-pilecom.py}
    \item allumer le mode radio ;
\pyline{radio.on()}
    \item lancer une boucle infinie et détecter si le \mb a été secoué ;
\pyfile{15}{16}{./res/mbpy-pilecom.py}
\end{enumerate}


Il faut maintenant ajouter les fonctions pour : 
\begin{enumerate}
    \item effectuer un tirage aléatoire d'un nombre entre 1 et 100 avec \pyline{random.randint(,)} et comparer ce nombre à la probabilité voulue.
    \item afficher \texttt{P} ou \texttt{F} \emph{selon l'issue du tirage} avec \pyline{dislay.show('')};
    \item envoyer la valeur 'P' ou 'F' par radio \emph{selon l'issue du tirage} avec \pyline{radio.send('')}
\end{enumerate}

Vérifie ton code avec   \includegraphics[width=0.1\linewidth]{res/check.png} et flash-le sur la carte avec \includegraphics[width=0.1\linewidth]{res/flash.png}


\end{eleve}


\newpage
\subsubsection{Notes pour l'enseignant}

Ce premier niveau permet de se familiariser avec l’interface de l'IDE en produisant un premier programme fonctionnel et utile.
Les notions de programmations utilisées sont:
\begin{enumerate}
    \item l'appel de fonction
    \item la boucle \pyline{while}
    \item la boucle \pyline{if}
\end{enumerate}

Dans cette première partie l'élève est aussi amené à concevoir une expérience aléatoire avec une probabilité donnée.

\begin{methode}
Pour résoudre ce problème, il suffit de programmer les instructions de la façon suivante : 
\pyfile{1}{5}{./res/mbpy-pilecom.py}
\pyfile{15}{17}{./res/mbpy-pilecom.py}
\pyfile{19}{25}{./res/mbpy-pilecom.py}
\end{methode}

\begin{remarque}
La variable \pyline{tirage} n'est pas indispensable, elle permet d'enregistrer le résultat du tirage aléatoire et de rendre le code plus lisible.

L'instruction \pyline{sleep(10)} permet de faire une pause de 10 ms dans le programme.

Afin de distinguer deux tirages successifs, on peut marquer une pause puis effacer l'écran à l'issue d'un tirage avec \pyline{display.clear()}, mais il peut-être amusant de montrer comment faire une animation simplement à partir d'une succession d'images.

Par exemple la fonction anim() ci-dessous pourra être définie au début du programme puis appeler avant chaque tirage.
\pyfile{14}{18}{./res/mbpy-pilecom.py}

\end{remarque}



%
% activité de niveau 2
%
\newpage
\subsection{Niveau Expert}
\subsubsection{Activité élève}

% commande perso \CARTOUCHE
%   5 paramètres : 
%       * durée
%       * public
%       * travail en maths
%       * travail en sciences
%       * travail en algo
\cartouche{0,5 h}{2de}{fréquences}{}{fonction ; événement ; variable}



\begin{eleve}
\texttt{\scshape{Mission : }
Réceptionne les données envoyées par les autres \mb et exploite-les pour tracer un \emph{graphique}}

Dans l'activité précédente nous avons créer une \emph{pièce virtuelle} qui affiche 'P' ou 'F' et qui envoie cette information par radio.
Nous allons programmer le \mb afin de recevoir, traiter et afficher les données. 


Le programme précédent devra être complété avec les éléments suivants :
\begin{enumerate}
    \item une déclaration de variables pour dénombrer les 'P' et les 'F', comme par exemple \pyline{p = 0}, qu'il faut écrire \emph{avant} la boucle \pyline{while True:}
    \item la réception d'un message radio (que l'on pourra enregistrer dans une variable 'issue') avec la fonction \pyline{radio.receive()} ;
    \item l'incrémentation des variables décomptant les 'P' et les 'F' par exemple \pyline{p = p + 1} ;
    \item le calcul de fréquence de 'P' et de 'F' (attention il faudra veiller à ne pas diviser par 0);
    \item l'affichage des fréquences dans le plotter avec \pyline{print((a,b))} où '(a,b)' est le couple de fréquences de 'P' et de 'F' dont on souhaite voir l'évolution au cours du temps.
\end{enumerate}



Vérifie ton code avec   \includegraphics[width=0.1\linewidth]{res/check.png} et flashe-le sur la carte avec \includegraphics[width=0.1\linewidth]{res/flash.png}.

Avant d'afficher le plotter avec \includegraphics[width=0.1\linewidth]{res/plotter.png}, il faudra peut-être faire un reset de la carte.  


\end{eleve}


\newpage
\subsubsection{Notes pour l'enseignant}

Cette deuxième activité nécessite impérativement l'usage de variables.

La difficulté peut provenir des différents cas de figure à traiter, mais à part l'instruction \pyline{else:}, il n'y a pas de nouveaux types d'instruction.

Les notions de programmations utilisées sont:
\begin{enumerate}
    \item la déclaration et l'initialisation de variables
    \item la boucle \pyline{if}, complété par \pyline{else}
\end{enumerate}

Dans cette partie l'élève est aussi amené à calculer des fréquences, mais l'exécution de ce calcul doit être conditionné à la non nullité de la somme des variables.

\begin{methode}
Pour recevoir et exploiter les données, il est possible de programmer les instructions de la façon suivante : \pyfile{1}{38}{./res/mbpy-pilecom.py}
\end{methode}
